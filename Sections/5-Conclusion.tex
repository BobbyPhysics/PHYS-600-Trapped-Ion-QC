\section{Conclusion}
As we've seen, there are many different types of ions used in trapped-ion quantum computing, and just as many different trapping architectures. Like most areas of quantum computing, it remains to be seen what technology will produce future scalable quantum platforms (if any of these do at all). It may be the case that a hybrid systems utilizing some aspects of point traps, QCCDs, and flying qubits all together will prove to be a successful scheme. In any case, ion crystals and RF traps are likely to remain a very important and promising path for quantum computing research.

2D Coulomb crystals in particular are a natural evolution of the linear ion chain, and one that this author presumes will offer the best chance at increasing scale beyond existing quantum processors. In any case, we are not confined to a one-dimensional path, and there is a whole world of ion traps to be explored.